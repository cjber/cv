\documentclass{scrartcl}
\usepackage[openbib]{currvita}
\usepackage[utf8]{inputenc}
\usepackage[bitstream-charter]{mathdesign}
\usepackage{fancyhdr}
\usepackage{lastpage}
\usepackage{hyperref}
\hypersetup{
    colorlinks=true,
    linkcolor=blue,
    filecolor=magenta,
    urlcolor=blue,
}
 
\urlstyle{same}

\setlength{\parindent}{0pt}


\newcommand*\ruleline[1]{\par\noindent\raisebox{.8ex}{\makebox[\linewidth]{\hrulefill\hspace{1ex}\raisebox{-.6ex}{#1}\hspace{1ex}}}}
\newcommand*\rulelinel[1]{\par\noindent\raisebox{.8ex}{\makebox[\linewidth]{\hrulefill\hspace{1ex}\raisebox{-.6ex}{#1}\hspace{1ex}\hrulefill}}}

\begin{document}

\pagestyle{fancy}
\fancyhf{}
\renewcommand{\headrulewidth}{0pt}
\renewcommand{\footrulewidth}{0pt}
\lhead{\rulelinel{\fontsize{9}{9}\bfseries Cillian Berragan \textbullet\ CV}}
\rfoot{\ruleline{\fontsize{9}{9}\bfseries Page \thepage\ of \pageref{LastPage}}}
\setlength{\footskip}{20pt}

\begin{cv}{Cillian Berragan}

\begin{cvlist}{Personal Details}
    \item cjberragan@gmail.com
    \item 07787796620
    \item \href{cjber.github.io}{cjber.github.io}
    \item \href{github.com/cjber}{github.com/cjber}
\end{cvlist}

    \small Graduate in \textit{Geographic Data Science} MSc from the University of Liverpool with a distinction (82\%). I am currently searching for a job that will broaden my analytical knowledge and skills, and build on my passion for data science.

\begin{cvlist}{Education}

\item[2018 - 2019] \normalsize\textit{MSc Geographic Data Science} (Distinction, 82\%): \textbf{University of Liverpool}

    \begin{quote}
        \small\textbf{Key Modules: }\textit{Geographic Data Science, Social Survey Analysis, Database and Information Systems}
    \end{quote}

    \small I utilised advanced techniques for geospatial data analysis in both R and Python, as well as worked with the relational database management system MySQL during this degree.

        \textbf{Dissertation:} \textit{Utilising Supervised Parametric Classification to Assess the Quality of the UK Rural Road Network using Aerial LiDAR Data.} (85\%)

    \texit{All assessments as part of this degree were awarded a distinction.}

    \item[2014 - 2017] \normalsize\textit{BSc (Hons) Coastal Marine Biology} (First-Class, 74\%): \textbf{University of Hull}

    \begin{quote}
        \small \textbf{Key Modules:} \textit{Independent Research Project, Geographic Information Systems, Environmental Impact Assessment}
    \end{quote}

    \small While this degree had a primary focus on coastal marine ecology it allowed me to develop a keen interest in statistical analysis, in particular through the use of the R software. I utilised R heavily in my undergraduate dissertation for which I received the prize for ‘Best Dissertation’ in my department.

\end{cvlist}

\begin{cvlist}{Employment}

    \item[2018] \normalsize\textit{Graduate Ecologist}: \textbf{TEP - The Environment Partnership}

    \small Daily tasks involved interacting with clients both over the phone and in person on site, preparing quotations for new ecological work and subsequently managing jobs that were accepted by clients. I worked with the ArcGIS software package to produce maps for professional use.

\end{cvlist}

\newpage
\begin{cvlist}{Key Skills}
    \small
    \item[\textbf{R:}] Statistical and geospatial analysis and visualisation primarily with ggplot2 and other tidyverse packages. Regression modelling using demographic data. Remote sensed data including both LiDAR and Aerial Imagery.
    \item[\textbf{Python:}] Statistical analysis, visualisation, machine learning techniques including $k$-means clustering, $k$-nearest neighbours and queen weighting spatial matrix's for geo-demographic classification and regionalisations.
    \item[\textbf{MySQL:}] Database querying and how this relates to relational algebra. As part of this module I produced a Boyce Codd normalized database from scratch.
    \item[\textbf{QGIS:}] At undergraduate level I developed an interest in QGIS in my spare time as an alternative to ArcGIS, covered as part of the degree. I have worked extensively in the past for both analysis and visualisation with this software but now focus more primarily on either Python or R. I produced a phase 1 habitat map for the Environment Agency for use in a habitat improvement project in the Tees Valley using QGIS.
    \item[\textbf{ArcGIS:}] As part of my undergraduate degree I had the opportunity to work with this software for both visualisation and analysis and worked professionally during my time at TEP.
    \item[\textbf{Writing:}] Produced a variety of academic written assignments for university and wrote professional environmental reports while working at TEP. A recent written assessment was awarded the highest grade in that module since the university program began (98\%). I am experienced with office products as well as all Linux equivalents, however primarily use R Markdown to produce most assignments.

        I can touch type at around 80+ wpm, and am very experienced using the \textit{hyper-extensible} text editor \textit{neovim}.

\end{cvlist}

    \small I have also developed an interest in GitHub, Linux and \LaTeX\ in my spare time, in particular to produce and host $R$ Markdown templates for university assessments, this CV was written entirely in \LaTeX\ which in particular I am working to become more familiar.

\begin{cvlist}{Interests}
    \item[]
    PADI Rescue Diver with Dry Suit Qualification, at the end of 2017 I travelled around South East Asia and Western Australia logging over 40 dives.

    I am interested in using R Markdown and \LaTeX\ to produce well formatted assessments for University and have produced an R Package containing various templates, hosted on my personal GitHub page.

    Very interested in computing in general, have built my own computer and custom keyboards. My text editor of choice, \textit{neovim}, is a particular interest of mine, I enjoy finding new methods for improving my work flow.

\end{cvlist}

\vfill

\small \textbf{Updated:} \end{cv}

\end{document}
