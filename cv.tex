\documentclass{scrartcl}
\usepackage[openbib]{currvita}
\usepackage[utf8]{inputenc}
\usepackage[bitstream-charter]{mathdesign}
\usepackage{fancyhdr}
\usepackage{lastpage}
\usepackage{hyperref}
\hypersetup{
    colorlinks=true,
    linkcolor=blue,
    filecolor=magenta,
    urlcolor=blue,
}
\usepackage{enumitem}
\setlist{nosep} 

 
\urlstyle{same}

\setlength{\parindent}{0pt}


\newcommand*\ruleline[1]{\par\noindent\raisebox{.8ex}{\makebox[\linewidth]{\hrulefill\hspace{1ex}\raisebox{-.6ex}{#1}\hspace{1ex}}}}
\newcommand*\rulelinel[1]{\par\noindent\raisebox{.8ex}{\makebox[\linewidth]{\hrulefill\hspace{1ex}\raisebox{-.6ex}{#1}\hspace{1ex}\hrulefill}}}

\begin{document}

\pagestyle{fancy}
\fancyhf{}
\renewcommand{\headrulewidth}{0pt}
\renewcommand{\footrulewidth}{0pt}
\lhead{\rulelinel{\fontsize{9}{9}\bfseries Cillian Berragan \textbullet\ CV}}
\rfoot{\ruleline{\fontsize{9}{9}\bfseries Page \thepage\ of \pageref{LastPage}}}
\setlength{\footskip}{20pt}

\begin{cv}{Cillian Berragan}

\begin{minipage}[t]{.6\textwidth}
\begin{cvlist}{Personal Details}
    \item cjberragan@gmail.com
    \item 07787796620
    \item \href{cjber.github.io}{cjber.github.io}
    \item \href{github.com/cjber}{github.com/cjber}
    \item \href{https://www.linkedin.com/in/cjberr/}{linkedin.com/in/cjberr/}
\end{cvlist}
\end{minipage}
    \begin{minipage}[t]{.4\textwidth}
        \footnotesize First class masters graduate in Geographic Data Science looking to move towards applied machine learning methods in business. I have a keen interest in python for machine learning, and am looking to build my existing knowledge.
    \end{minipage}


\begin{cvlist}{Key Skills}
\small
\item[\textbf{Python:}]\href{https://cjber.github.io/tags/python/}{\textbf{Python Projects}}
\begin{itemize}\item Machine learning; \texttt{spaCy, PyTorch, sklearn}; familiar with deep learning pipelines and mathematical foundations. 
\item Data Analysis; \texttt{pandas, numpy, geopandas}.
\item Data Visualisation; \texttt{matplotlib, seaborn}.
\item Object Orientated Programming.
\end{itemize}

\vspace{5mm}
\item[\textbf{R:}]\href{https://cjber.github.io/tags/r/}{\textbf{R Projects}}
\begin{itemize}\item \texttt{tidyverse}; Analysis, Preprocessing, Statistics.
\item Data Visualisation; \texttt{ggplot2, sf}.
\item R Markdown Notebooks; Reporting results in a readable format.

\end{itemize}

\vspace{5mm}
\item[\textbf{SQL:}]\begin{itemize}\item Database querying and relational algebra.
\item Database construction.
\vspace{5mm}

\end{itemize}

% \item[\textbf{QGIS \& ArcGIS:}]\begin{itemize}\item Client ready visualisations.
% \item Survey map produced for the Environment Agency using QGIS.
% \item ArcMap Used professionally at TEP.

% \end{itemize}

\vspace{5mm}
\item[\textbf{Written work:}]\begin{itemize}\item Academic and professional reports, all at distinction standard during MSc.
\item One written assessment during my MSc was awarded the highest grade in that module since the university program began (98\%).
\item Primarily use the Linux operating system with R Markdown or \LaTeX\ for written work in the Neovim text editor. My \href{https://github.com/cjber/dotfiles}{dotfiles} contain my particular workflow configurations. 
\item Touch type at 80+ wpm.
\vspace{5mm}
\item[\textbf{Additional Skills}:] Regex, NLP, \LaTeX, R Markdown, Linux, Bash, Vim, Git

\end{itemize}
\end{cvlist}

\newpage
\begin{cvlist}{Education}

\item[2019 - present] \normalsize\textit{PhD Student, Data Analytics and Society}: \textbf{University of Liverpool}

    \begin{quote}
        \small \textit{Improving the Geolocation of Emergency Service Response through Big Data}
    \end{quote}

    \small Advanced NLP with spaCy package. Machine Learning in Python, PyTorch implemented neural networks. Foundational knowledge of deep learning with a focus on NLP. Collaborative work as part of a research lab.

\item[2018 - 2019] \normalsize\textit{MSc Geographic Data Science} (Distinction, 82\%): \textbf{University of Liverpool}

    \small Utilised machine learning techniques for geospatial data analysis in both R and Python, and worked with the relational database management system MySQL. All assessments as part of this degree were awarded a distinction (70+\%). I recieved the highest overall mark in my programme, and my dissertation was nominated for the RGS Geographical Information Science Research Group (GIScRG) award.

        \textbf{Dissertation:} \textit{\href{https://github.com/cjber/dissertation/blob/master/201374125.pdf}{Utilising Supervised Parametric Classification to Assess the Quality of the UK Rural Road Network using Aerial LiDAR Data.}} (85\%)
    \item[2014 - 2017] \normalsize\textit{BSc (Hons) Coastal Marine Biology} (First-Class, 74\%): \textbf{University of Hull}

    \small While this degree had a primary focus on coastal marine ecology it allowed me to develop a keen interest in statistical analysis, in particular through the use of the R software. I utilised R heavily in my undergraduate dissertation for which I received the prize for ‘Best Dissertation’ in my department.

\end{cvlist}

\begin{cvlist}{Employment}

    \item[2018] \normalsize\textit{Graduate Ecologist}: \textbf{TEP - The Environment Partnership}

    \small Daily tasks involved interacting with clients both over the phone and in person on site, preparing quotations for new ecological work and subsequently managing jobs that were accepted by clients. Through this work I gained confidence in a business environment, and developed my leadership and teamwork skills.

\end{cvlist}


\begin{cvlist}{Additional Interests}
\item[]

    \small I am interested in the Linux operating system and personally use Arch Linux with the i3 window manager for my work. I strongly believe in reproducibility in any work, and use R Markdown and \LaTeX\ to produce well formatted, compiled assessments for University, with my themes hosted on my personal \href{https://github.com/cjber/uolrmarkdown}{GitHub page}. I am also passionate about contributing towards the open source programs I use and make pull requests when I am able to.


\end{cvlist}

\vfill

\small \textbf{Updated:} \end{cv}

\end{document}
